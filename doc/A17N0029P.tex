\documentclass[a4paper]{article}
\usepackage{pdflscape}
\usepackage{fancyvrb}
\usepackage{soul}
\usepackage{url}
\usepackage{graphicx}
\usepackage{xcolor}
\usepackage{listings}
\usepackage{amsmath}
\usepackage[czech]{babel}
\definecolor{cervena}{rgb}{1,0,0}
\definecolor{seda}{rgb}{.9,.9,.9}
\definecolor{zelena}{rgb}{0,.5,0}
\definecolor{modra}{rgb}{0,0,1} 
\definecolor{jina}{rgb}{.3,0,0} 
\lstset{
	numberstyle=tiny,
	basicstyle=\ttfamily,
	keywordstyle=\color{modra}\bf\tt,
	stringstyle=\color{cervena},
	backgroundcolor=\color{seda},
	captionpos=b,
	breaklines=true,
	tabsize=2,
	frame=single,
	escapechar=$
}
\begin{document}
\begin{titlepage}
	\includegraphics[width=0.5\textwidth]{logo.pdf}
	\par
	\centering
	\vspace{2cm}
	{\scshape\Huge P\v{r}eklada\v{c} PL/0 do x86\par}
		\vspace{1cm}
	{\scshape\large KIV/FJP - Semestr\'aln\'i pr\'ace\par}
	\vfill
	\raggedleft
	\begin{table}[!ht]
		\begin{tabular}{l l l}
			t\'ym: & x86ka\v{r}i\\
			studenti: & Martin \v{C}ervenka & \st{Zden\v{e}k Ziegler}\\
			studijn\'i \v{c}\'isla: & A17N0029P & \st{A18N0104P}\\
			emaily: & cervemar@students.zcu.cz & \st{zieglerz@students.zcu.cz}\\
			odevzd\'an\'i: & \today\\
			github: & \multicolumn{2}{l}
				{\url{https://github.com/cervenkam/fjp}}
		\end{tabular}
	\end{table}
\end{titlepage}
\tableofcontents
\clearpage
\section{Zad\'an\'i}
C\'ilem pr\'ace bude vytvo\v{r}en\'i p\v{r}eklada\v{c}e zvolen\'eho jazyka.
Je mo\v{z}n\'e inspirovat se jazykem PL/0, vybrat si podmno\v{z}inu
n\v{e}jak\'eho existuj\'ic\'iho jazyka nebo si navrhnout jazyk zcela vlastn\'i.
D\'ale je tak\'e pot\v{r}eba zvolit si pro jakou architekturu bude jazyk
p\v{r}ekl\'ad\'an (doporu\v{c}eny jsou instrukce PL/0, ale je mo\v{z}n\'e
zvolit jakoukoliv instruk\v{c}n\'i sadu pro kterou budete m\'it interpret). 

Jazyk mus\'i m\'it minim\'aln\v{e} n\'asleduj\'ic\'i konstrukce:

\begin{itemize}
\item definice celo\v{c}\'iseln\'ych prom\v{e}nn\'ych
\item definice celo\v{c}\'iseln\'ych konstant
\item p\v{r}i\v{r}azen\'i
\item z\'akladn\'i aritmetiku a logiku (\texttt{+},\texttt{-},\texttt{*},
	\texttt{/}, \texttt{AND}, \texttt{OR}, negace a z\'avorky,
	oper\'atory pro porovn\'an\'i \v{c}\'isel)
\item cyklus (libovoln\'y)
\item jednoduchou podm\'inku (if bez else)
\item definice podprogramu (procedura, funkce, metoda) a jeho vol\'an\'i
\end{itemize}
P\v{r}eklada\v{c} kter\'y bude um\v{e}t tyto z\'akladn\'i v\v{e}ci bude hodnocen
deseti body. Dal\v{s}\'i body (alespo\v{n} do minim\'aln\'ich 20) je mo\v{z}n\'e
z\'iskat na z\'aklad\v{e} roz\v{s}\'i\v{r}en\'i, jsou rozd\v{e}leny do dvou
skupin, jednodu\v{s}\v{s}\'i za jeden bod a slo\v{z}it\v{e}j\v{s}\'i za dva
a\v{z} t\v{r}i body. Dal\v{s}\'i roz\v{s}\'i\v{r}en\'i je mo\v{z}no doplnit po
konzultaci, s ohodnocen\'im podle odhadnut\'e n\'aro\v{c}nosti. 

Jednoduch\'a roz\v{s}\'i\v{r}en\'i (1 bod):

\begin{itemize}
\item ka\v{z}d\'y dal\v{s}\'i typ cyklu (\texttt{for}, \texttt{do ... while},
	\texttt{while ... do}, \texttt{repeat ... until}, \texttt{foreach} pro pole)
\item \texttt{else} v\v{e}tev
\item datov\'y typ \texttt{boolean} a logick\'e operace s n\'im
\item datov\'y typ \texttt{real} (s celo\v{c}\'iseln\'ymi instrukcemi)
\item datov\'y typ \texttt{string} (s oper\'atory pro spojov\'an\'i
	\v{r}\v{e}tezc\r{u})
\item rozv\v{e}tven\'a podm\'inka (\texttt{switch}, \texttt{case})
\item n\'asobn\'e p\v{r}i\v{r}azen\'i (\texttt{a = b = c = d = 3;})
\item podm\'in\v{e}n\'e p\v{r}i\v{r}azen\'i / tern\'arn\'i oper\'ator
	(\texttt{min = (a < b) ? a : b;})
\item paraleln\'i p\v{r}i\v{r}azen\'i (\texttt{\{a, b, c, d} = {1, 2, 3, 4\};})
\item p\v{r}\'ikazy pro vstup a v\'ystup (\texttt{read}, \texttt{write} --
	pot\v{r}ebuje vhodn\'e instrukce kter\'e bude mo\v{z}n\'e vyu\v{z}\'it)
\end{itemize}
\clearpage
slo\v{z}it\v{e}j\v{s}\'i roz\v{s}\'i\v{r}en\'i (2 body):

\begin{itemize}
\item p\v{r}\'ikaz \texttt{GOTO} (pozor na vzd\'alen\'e skoky)
\item datov\'y typ \texttt{ratio} (s celo\v{c}\'iseln\'ymi instrukcemi)
\item slo\v{z}en\'y datov\'y typ (\texttt{Record})
\item pole a pr\'ace s jeho prvky
\item oper\'ator pro porovn\'an\'i \v{r}et\v{e}zc\r{u}
\item parametry p\v{r}ed\'avan\'e hodnotou
\item n\'avratov\'a hodnota podprogramu
\item objekty bez polymorfismu
\item anonymn\'i vnit\v{r}n\'i funkce (lambda v\'yrazy)
\end{itemize}

Roz\v{s}\'i\v{r}en\'i vy\v{z}aduj\'ic\'i slo\v{z}it\v{e}j\v{s}\'i
instruk\v{c}n\'i sadu ne\v{z} m\'a PL/0 (3 body):

\begin{itemize}
\item dynamicky p\v{r}i\v{r}azovan\'a pam\v{e}\v{t} - pr\'ace s ukazateli
\item parametry p\v{r}ed\'avan\'e odkazem
\item objektov\'e konstrukce s polymorfní\'im chov\'an\'im
\item \texttt{instanceof} oper\'ator
\item anonymn\'i vnit\v{r}n\'i funkce (lambda v\'yrazy) kter\'e lze p\v{r}edat
	jako parametr
\item mechanismus zpracov\'an\'i v\'yjimek
\end{itemize}

Vlastn\'i interpret (\v{r}\'adkov\'y, bez rozhran\'i, slo\v{z}it\'y alespo\v{n}
jako roz\v{s}\'i\v{r}en\'a PL/0) je za 6 bod\r{u}. 
\clearpage
\section{Teoretick\'y \'uvod}
P\v{r}eklada\v{c}e jsou ned\'ilnou sou\v{c}\'ast\'i dne\v{s}n\'ich
po\v{c}\'ita\v{c}\r{u}. Mezi p\v{r}eklada\v{c}e lze \v{r}adit cokoliv,
do \v{c}eho vstupuje n\v{e}jak\'y p\v{r}edem definovan\'y form\'at a na
v\'ystupu je jin\'y, takt\'e\v{z} definovan\'y form\'at.
Dne\v{s}n\'i ch\'ap\'an\'i pojmu p\v{r}eklada\v{c} se v\v{s}ak posunulo
sm\v{e}rem k p\v{r}ekladu programovac\'ich jazyk\r{u}, a\v{t} u\v{z} mezi
sebou nebo nap\v{r}\'iklad do strojov\'eho k\'odu procesoru.

\subsection{Jazyk PL/0}
Jazyk PL/0 v dne\v{s}n\'i dob\v{e} nen\'i tak roz\v{s}\'i\v{r}en, proto\v{z}e
je p\v{r}\'ili\v{s} jednoduch\'y pro re\'aln\'e pou\v{z}it\'i, av\v{s}ak
vych\'az\'i z jazyku Pascal a je vhodn\'y
pro vytvo\v{r}en\'i jednoduch\'eho p\v{r}eklada\v{c}e.

Tento jazyk disponuje pouze jednoduchou celo\v{c}\'iselnou
aritmetikou, jednoduch\'ym v\v{e}tven\'im a jedn\'im druhem cyklu.
Oproti dne\v{s}n\'im b\v{e}\v{z}n\'ym jazyk\r{u}m (C-like, Java atd.)
ale naopak podporuje plnohodnotn\'e zano\v{r}en\'e procedury.

Jazyk nen\'i s\'am o sob\v{e} pou\v{z}iteln\'y, proto\v{z}e nem\'a
\v{z}\'adn\'y apar\'at pro vstup a v\'ystup ani z konzole ani
ze souboru ani odjinud. To znamen\'a, \v{z}e bude n\v{e}co
po\v{c}\'itat, co mus\'ime zadat p\v{r}i p\v{r}ekladu a v\'ysledek
nem\'ame mo\v{z}nost b\v{e}\v{z}n\'ymi prost\v{r}edky z\'iskat. Proto
je nutn\'e tento jazyk roz\v{s}\'i\v{r}it.
\subsection{Procesor x86-64}
Dnes jedn\'im z velmi pou\v{z}\'ivan\'ych procesor\r{u} v osobn\'ich
po\v{c}\'ita\v{c}\'ich je architektury x86-64. Tato architektura je dnes
a\v{z} 64bitov\'a, ale je zp\v{e}tn\v{e} kompatibiln\'i s 32bitov\'ym
a 16bitov\'ym re\v{z}imem.

Prav\v{e} v 16bitov\'em re\v{z}imu procesor startuje. V tomto re\v{z}imu
dok\'a\v{z}e teoreticky adresovat pouze 64kB pam\v{e}ti, ale d\'iky
segmentaci m\'a vlastn\v{e} adresaci 20bity, tedy 1MB pam\v{e}ti. Do
n\v{e}kter\'ych \v{c}\'ast\'i t\'eto segmentovan\'e pam\v{e}ti se na\v{c}te
nap\v{r}. BIOS, nebo se rezervuje pro video pam\v{e}\v{t}, co\v{z} redukuje
rozsah pou\v{z}iteln\'e pam\v{e}ti na 640kB\footnote{640KB
ought to be enough for anybody.}. Tento re\v{z}im byl d\v{r}\'ive
p\v{r}i p\v{r}edstaven\'i i8086 jedin\'ym, s n\'astupem nov\v{e}j\v{s}\'i
procesor\r{u} s chr\'an\v{e}n\'ym re\v{z}imem se tomuto 16bitov\'emu
re\v{z}imu za\v{c}alo \v{r}\'ikat re\'aln\'y.

Tato pr\'ace se konkr\'etn\v{e} bude zab\'yvat p\v{r}eklada\v{c}em z
jazyka PL/0 do strojov\'eho k\'odu instrukc\'i dnes b\v{e}\v{z}n\v{e}
pou\v{z}\'ivan\'eho procesoru x86 v 16bitov\'em re\'al\-n\'em re\v{z}imu.
\clearpage
\section{Platforma}
U p\v{r}eklada\v{c}e se musej\'i definovat dv\v{e} platformy, jedna je ta
zdrojov\'a, tedy v jak\'em programovac\'im jazyce se program programuje,
a druh\'a je c\'ilov\'a, tedy pro jak\'y syst\'em se bude program
p\v{r}ekl\'adat.
\subsection{Zdrojov\'a platforma}
P\v{r}eklada\v{c} je naps\'an v programovac\'im jazyce Java, proto je
kompatibiln\'i jak\'eko\-liv za\v{r}\'izen\'i, kde b\v{e}\v{z}\'i Java
Virtual Machine (JVM) ve verzi alespo\v{n} 1.8.
\subsection{Zdrojov\'y program}
Jako zdrojov\'y (p\v{r}ekl\'adan\'y) programovac\'i jazyk byl zvolen jazyk
PL/0, kter\'y byl roz\v{s}\'i\v{r}en oproti p\r{u}vodn\'i verzi.
Jedn\'a se konkr\'etn\v{e} o tato roz\v{s}\'i\v{r}en\'i:

\begin{itemize}
\item v\v{e}tev \texttt{else}
\item p\v{r}\'ikaz \texttt{read} a \texttt{write} pro \v{c}ten\'i/z\'apis z/do
	konzole
\item p\v{r}\'ikaz \texttt{execute}, kter\'y vykon\'a konkr\'etn\'i
	zadan\'y strojov\'y k\'od (podrobnosti d\'ale)
\item lok\'aln\'i \texttt{goto} skoky na zadan\'a n\'av\v{e}\v{s}t\'i
	(\texttt{label})
\item jedno\v{r}\'adkov\'e koment\'a\v{r}e za\v{c}\'inaj\'ic\'i znaky
	\texttt{\textbackslash\textbackslash}
\item logick\'e operace (\texttt{AND}, \texttt{OR}, \texttt{XOR}, \texttt{NOT})
	a bitov\'e posuvy (\texttt{LEFT}, \texttt{RIGHT}), kter\'e takt\'e\v{z}
	v p\r{u}vodn\'i PL/0 chyb\'i
\end{itemize}

\noindent
Zjednodu\v{s}en\v{e} zapsan\'a gramatika je tak\'e uvedena v p\v{r}\'iloze
\ref{app:gram}.
\subsection{C\'ilov\'a platforma}
Pro c\'ilovou platformu jsem zvolil instrukce procesoru x86. Co se t\'y\v{c}e
softwarov\'e \v{c}\'asti, tak se program p\v{r}ekl\'ad\'a do bootovateln\'e
podoby, proto nepot\v{r}ebuje \v{z}\'adn\'y opera\v{c}n\'i syst\'em, jedin\'y
SW kter\'y v z\'akladn\'i verzi pot\v{r}ebuje, je vol\'an\'i BIOSu pro
\v{c}ten\'i a z\'apis z konzole.
\clearpage
\section{U\v{z}ivatelsk\'a p\v{r}\'iru\v{c}ka}\label{sec:uziv}
V t\'eto sekci je uvedna z\'akladn\'i manipulace s p\v{r}eklada\v{c}em.
\subsection{Prerekvizity}
Spu\v{s}t\v{e}n\'i je mo\v{z}n\'e prov\'est d\v{e}ma zp\r{u}soby.
Doporu\v{c}en\'ym zp\r{u}sobem je pou\v{z}it\'i emul\'atoru (nap\v{r} emul\'ator
\textbf{QEMU}), kde prakticky nem\r{u}\v{z}e doj\'it k tomu, aby
p\v{r}elo\v{z}en\'y program provedl ne\v{z}\'adanou operaci typu
maz\'an\'i/p\v{r}episov\'an\'i soubor\r{u} (nap\v{r} syst\'emov\'ych). Druh\'y
zp\r{u}sob (na vlastn\'i riziko) je spu\v{s}t\v{e}n\'i p\v{r}\'imo na HW.
Ob\v{e} mo\v{z}nosti budou diskutov\'any.
\subsection{OS Linux a emul\'ator QEMU}
P\v{r}ekl\'ad\'a-li se program na OS Linux, jsou dod\'any skripty pro
p\v{r}eklad a spu\v{s}t\v{e}n\'i.
\subsubsection{P\v{r}eklad programu}
Pro p\v{r}eklad se pou\v{z}\'iv\'a skript \texttt{compile.sh} a m\'a dva
nepovinn\'e parametry. Prvn\'im je cesta k p\v{r}ekl\'adan\'emu programu a
m\'a p\v{r}ednastavenou hodntotu \texttt{hilbert.pl0}. Druh\'ym je cesta
k vyprodukovan\'emu bin\'arn\'imu souboru -- p\v{r}ed\-nastaveno je
\texttt{out.bin}. N\'asleduj\'ic\'i dva p\v{r}\'ikazy jsou tedy ekvivalentn\'i:
\begin{verbatim}
./compile.sh
./compile.sh hilbert.pl0 out.bin
\end{verbatim}
\subsubsection{Spu\v{s}t\v{e}n\'i}
Pro spu\v{s}t\v{e}n\'i je p\v{r}ipraven skript \texttt{run.sh} a m\'a jeden
nepovinn\'y parametr, kter\'ym je p\v{r}elo\v{z}en\'y soubor (s defaultn\'i
hodnotou \texttt{out.bin}. N\'asleduj\'ic\'i dva p\v{r}\'ikazy jsou tedy
op\v{e}t ekvivalentn\'i:
\begin{verbatim}
./run.sh
./run.sh out.bin
\end{verbatim}
Spust\'i se okno emul\'atoru a program se za\v{c}ne vykon\'avat. Je-li
v programu p\v{r}\'ikaz \texttt{read}, tedy je-li vy\'v{z}adov\'an vstup
od u\v{z}ivatele, tak ten se zad\'av\'a v bin\'arn\'i reprezentaci
bez prefixu, tedy pouze \texttt{0} a \texttt{1}, jak\'akoliv jin\'a kl\'avesa
ukon\v{c}\'i vstup. Vol\'an\'i v\'ypisu (\texttt{write}) tatk\'e\v{z}
vypisuje v bin\'arn\'i podob\v{e}.
\subsection{Jin\'y OS a emul\'ator QEMU}
Pokud je program spou\v{s}t\v{e}n na jin\'em OS ne\v{z} linux, pak bude
pravd\v{e}podobn\v{e} nutn\'e spustit p\v{r}eklad a program jin\'ym
zp\r{u}sobem.

Jednoduchou cestou lze postupovat jako v p\v{r}\'ipad\v{e} linuxu, akor\'at
drobn\v{e} upravit skripty, aby na dan\'em OS fungovaly. Oba skripty byly
ps\'any tak, aby byly co nejjednodu\v{s}\v{s}\'i, tak\v{z}e ka\v{z}d\'y
z t\v{e}ch skript\r{u} obsahuje pouze jeden p\v{r}\'ikaz, kter\'y lze
tedy zadat p\v{r}\'imo do konzole nebo ho takt\'e\v{z} upravit.

P\v{r}\'ikazy, kter\'e by se musely upravit jsou:
\begin{verbatim}
java -jar compiler.jar $1 $2 
qemu-system-i386 -drive file=${1:-out.bin},format=raw
\end{verbatim}
\subsection{Spu\v{s}t\v{e}n\'i p\v{r}\'imo na HW}
Tento krok nemohu doporu\v{c}it, proto\v{z}e program bude spu\v{s}t\v{e}n
v tzv. re\'aln\'em m\'odu a m\r{u}\v{z}e teoreticky po\v{s}kodit
n\v{e}jak\'e SW vybaven\'i po\v{c}\'ita\v{c}e. P\v{r}eklad programu se
provede stejn\v{e} jako je uvedeno v\'y\v{s}e, postup se li\v{s}\'i a\v{z}
v p\v{r}\'ipad\v{e} spu\v{s}t\v{e}n\'i.

Program se nakop\'iruje na za\v{c}\'atek
pam\v{e}ti bootovateln\'eho m\'edia (nap\v{r}. flash nebo HDD), toho
lze v linuxu dos\'ahnout nap\v{r}. p\v{r}\'ikazem \texttt{dd}. V BIOSu se
p\v{r}\'ipad\-n\v{e} nastav\'i bootov\'an\'i z toho media a n\'asledn\v{e}
po\v{c}\'ita\v{c} p\v{r}i bootov\'an\'i program spust\'i. V\'yhoda tohoto
postupu je v tom, \v{z}e program nebude emulov\'an a bude se vykon\'avat
rychleji. Nev\'yhodou je slo\v{z}it\v{e}j\v{s}\'i postup spu\v{s}t\v{e}n\'i a
hlavn\v{e} nebezpe\v{c}\'i po\v{s}kozen\'i SW dan\'eho po\v{c}\'ita\v{c}e.
Tento postup by m\v{e}l prov\'est pouze pokro\v{c}il\'y u\v{z}ivatel se
znalost\'i p\v{r}ekl\'adan\'eho programu.
\clearpage
\section{Struktura projektu}
Projekt krom\v{e} tohoto souboru obsahuje p\v{e}t slo\v{z}ek, bashovs\'e
skripty \texttt{*.sh},
testovac\'i p\v{r}ekl\'adan\'e soubory \texttt{*.pl0}, p\v{r}elo\v{z}en\'y
p\v{r}eklada\v{c} \texttt{compiler.jar},
ANTLR jazykov\'y processor \texttt{antlr.jar}, a markdownov\'y soubor
\texttt{README.md}.
\subsection{Podslo\v{z}ky projektu}
Projekt obsahuje p\v{e}t podslo\v{z}ek. Prvn\'i slo\v{z}ka je \texttt{bin},
kde jsou ulo\v{z}eny p\v{r}elo\v{z}en\'e programy. Druh\'a slo\v{z}ka
\texttt{compiler} obsahuje zdrojov\'e soubory p\v{r}eklada\v{c}e v jazyce
Java. T\v{r}et\'i podslo\v{z}kou je \texttt{kernel}, kde je zdrojov\'y soubor
zav\'ad\v{e}n\'eho j\'adra v jazyce symbolick\'ych adres, kter\'y program
spou\v{s}t\'i. D\'ale n\'asleduje slo\v{z}ka \texttt{pictures} se sn\'imky
b\v{e}hu testovac\'ich program\r{u} a nakonec slo\v{z}ka \texttt{shouldfail}
obsahuje zdrojov\'e soubory, kter\'e jsou sice syntakticky spr\'av\-n\v{e},
ale nejsou spr\'av\-n\v{e} s\'emanticky a nem\v{e}lo by mo\v{z}n\'e je
p\v{r}elo\v{z}it, resp. p\v{r}eklad se ukon\v{c}\'i chybou.
\subsection{Bash skripty}
V rootu projektu se nach\'az\'i p\v{e}t bashovs\'ych skript\r{u}. Dva z nich
\texttt{compile.sh} a \texttt{run.sh} jsou ji\v{z} pops\'any v
sekci \ref{sec:uziv}. Dal\v{s}\'im je skript \texttt{build.sh}. Ten se
mus\'i spustit v p\v{r}\'ipad\v{e}, \v{z}e se m\v{e}n\'i zdrojov\'e
soubory p\v{r}eklada\v{c}e (\texttt{*.java}) nebo j\'adra (\texttt{loader.asm}).
Dal\v{s}\'im skriptem je \texttt{debug.sh}, kter\'y vyp\'i\v{s}e v
jazyce symbolick\'ych adres, co vlastn\v{e} program d\v{e}l\'a.
Posledn\'im je \texttt{clean.sh}, kter\'y ma\v{z}e soubory kter\'e
se generuj\'i p\v{r}i p\v{r}ekladu.
\subsection{P\v{r}ekl\'adan\'e programy}
Projekt dod\'av\'a p\v{e}t p\v{r}edp\v{r}ipraven\'ych testovac\'ich zdrojov\'ych
soubor\r{u} s koncovkou \texttt{pl0}. V\'ystup t\v{e}ch
nejzaj\'imav\v{e}j\v{s}\'ich je zn\'azorn\v{e}n na obr\'azc\'ich
\ref{obr:hilbert} a \ref{obr:circle} v p\v{r}\'iloze \ref{app:obr}, dal\v{s}\'i
obr\'azky jsou ve slo\v{z}ce \texttt{pictures}.
\subsection{P\v{r}eklada\v{c}}
Spustiteln\'y soubor p\v{r}eklada\v{c}e je soubor \texttt{compiler.jar}. K tomu
aby fungoval pot\v{r}ebuje m\'it ve stejn\'e pracovn\'i slo\v{z}ce i soubor
\texttt{antlr.jar}
\subsection{Veden\'i projektu}
Projekt je verzov\'an programem \texttt{git} a veden na webov\'e str\'ance
Githubu na adrese \url{https://github.com/cervenkam/fjp}. 
\clearpage
\section{P\v{r}\'iklad p\v{r}ekladu}
M\v{e}jme minim\'aln\'i program uveden\'y v \ref{program}.
Program pouze alokuje na z\'asobn\'iku jednu prom\v{e}nnou, p\v{r}i\v{r}ad\'i
j\'i hodnotu 3 a tuto hodnotu tak\'e vyp\'i\v{s}e.

\begin{figure}[!ht]
\centering
\begin{Verbatim}[commandchars=\\\{\}]
\textcolor{red}{var x;}
\textcolor{blue}{begin}
	\textcolor{zelena}{x:=3;}
	\textcolor{jina}{write x;}
\textcolor{magenta}{end.}
\end{Verbatim}
\caption{P\v{r}ekl\'adan\'y program}
\label{program}
\end{figure}

P\v{r}elo\v{z}en\'im
z\'isk\'ame strojov\'y k\'od, kter\'y kdy\v{z} p\v{r}evedeme do jazyka
symbolick\'ych adres, bude vypadat jak je uvedeno v \ref{objdump}.

\begin{figure}[!ht]
\centering
\begin{Verbatim}[commandchars=\\\{\}]
00000200 <.data+0x200>:
 200: \textcolor{blue}{66 67 c8} \textcolor{red}{04 00} \textcolor{blue}{01}    \textcolor{blue}{addr32 enter} \textcolor{red}{0x4}\textcolor{blue}{,0x1}
 206: \textcolor{blue}{66 e9 00 00 00 00    jmp    0x20c}
 20c: \textcolor{zelena}{66 89 e8             mov    eax,ebp}
 20f: \textcolor{zelena}{66 2d 08 00 00 00    sub    eax,0x8}
 215: \textcolor{zelena}{66 67 50             addr32 push eax}
 218: \textcolor{zelena}{66 67 68 03 00 00 00 addr32 push 0x3}
 21f: \textcolor{zelena}{66 67 58             addr32 pop eax}
 222: \textcolor{zelena}{66 67 5b             addr32 pop ebx}
 225: \textcolor{zelena}{66 67 36 89 03       mov    DWORD PTR ss:[ebx],eax}
 22a: \textcolor{jina}{66 89 e8             mov    eax,ebp}
 22d: \textcolor{jina}{66 2d 08 00 00 00    sub    eax,0x8}
 233: \textcolor{jina}{66 67 50             addr32 push eax}
 236: \textcolor{jina}{66 67 5b             addr32 pop ebx}
 239: \textcolor{jina}{66 67 36 8b 03       mov    eax,DWORD PTR ss:[ebx]}
 23e: \textcolor{jina}{9a 80 01 c0 07       call   0x7c0:0x180}
 243: \textcolor{magenta}{66 67 c9             addr32 leave}
 246: \textcolor{magenta}{eb fe                jmp    0x246}
\end{Verbatim}
\caption{P\v{r}elo\v{z}en\'y program, vygenerov\'ano pomoc\'i \texttt{debug.sh}}
\label{objdump}
\end{figure}

Prom\v{e}nn\'a se alokuje instrukc\'i \texttt{enter}, kter\'a pro 32bitovou
prom\v{e}nnou alokuje \texttt{0x4} bajty. Tato prom\v{e}nn\'a bude ulo\v{z}ena
na z\'asobn\'iku na pozici \texttt{ebp-0x8}. Aby se prom\v{e}nn\'a nastavila na
hodnotu 3, nejprve se na\v{c}te jej\'i adresa a vlo\v{z}\'i se do z\'asobn\'iku
(prvn\'i t\v{r}i zelen\'e instrukce), pot\'e se do z\'asobn\'iku vlo\v{z}\'i i
p\v{r}i\v{r}azovan\'a hodnota, tedy \texttt{0x3}, nakonec se hodnota i adresa
ze z\'asobn\'iku vyberou a hodnota se na adresu vlo\v{z}\'i.

V dal\v{s}\'im p\v{r}\'ikazu v\'ypisu se op\v{e}t na\v{c}te adresa
prom\v{e}nn\'e \texttt{x} (prvn\'i t\v{r}i tmav\v{e} \v{c}erven\'e instrukce),
pot\'e se vrchol z\'asobn\'iku dereferencuje a vlo\v{z}\'i do registru
\texttt{eax} (n\'asleduj\'ic\'i dv\v{e} instrukce) a nakonec se zavol\'a
v\'ypis, o kter\'y se star\'a j\'adro a vypisuje hodnotu registru \texttt{eax}.

Nakonec se dealokuje prom\v{e}nn\'a \texttt{x} a program je \uv{ukon\v{c}en}
zacyklen\'im na jedn\'e instrukci.

\section{Z\'av\v{e}r}
Co se t\'y\v{c}e negativn\'ich vlastnost\'i, p\v{r}ekl\'adan\'y jazyk
nedisponuje formou p\v{r}\'im\'eho p\v{r}ed\'av\'an\'i parametr\r{u}.
To je sice mo\v{z}n\'e obejt\'it pomoc\'i glob\'aln\'ich a lok\'aln\'ich
prom\v{e}nn\'ych, ale d\v{e}l\'a to k\'od ne\v{c}iteln\'ym (viz program
\texttt{hilbert.pl0}). P\v{r}eklada\v{c} takt\'e\v{z} neposkytuje
z\'avratn\'e optimalizace. V\v{e}t\v{s}inu \v{c}asu pouze vkl\'ad\'a
a vyb\'ir\'a ze z\'asobn\'iku m\'isto p\v{r}\'im\'eho v\'ypo\v{c}tu v
registrech.

I p\v{r}esto zde p\v{r}edstaven\'y p\v{r}eklada\v{c} dok\'a\v{z}e
p\v{r}elo\v{z}it i slo\v{z}it\v{e}j\v{s}\'i program, nap\v{r}\'i\-klad zde
uveden\'y rekurzivn\'i v\'ypo\v{c}et a vykreslen\'i Hilbertovy k\v{r}ivky.
Jazyk PL/0 byl roz\v{s}\'i\v{r}en o n\v{e}kolik syntaktick\'ych v\'yraz\r{u}.
P\v{r}eklada\v{c} byl otestov\'an na n\v{e}kolika programech s pozitivn\'imi
v\'ysledky a byl t\'e\v{z} testov\'an na programy, kter\'e jsou s\'emanticky
chybn\'e, takt\'e\v{z} \'usp\v{e}\v{s}n\v{e}. Program nen\'i z\'avisl\'y
na \v{z}\'adn\'em SW (krom\v{e} BIOSu) a je mo\v{z}n\'e ho spustit
na libovoln\'em procesoru architekturou kompatibiln\'i s i386.

\section{Reference}
\begin{itemize}
\item Instrukce 80386: \url{https://pdos.csail.mit.edu/6.828/2007/readings/i386/c17.htm}
\item Instrukce x86: \url{https://www.felixcloutier.com/x86/}
\item Dokumentace ANTLR: \url{https://www.antlr.org/api/Java/index.html}
\item P\v{r}eru\v{s}en\'i BIOSu: \url{http://www.ctyme.com/intr/int.htm}
\item J\'adro OS: \url{https://github.com/cervenkam/os}
\item Github projektu: \url{https://github.com/cervenkam/fjp}
\end{itemize}
\clearpage
\appendix
\begin{landscape}
\section{Gramatika jazyka}\label{app:gram}
\begin{lstlisting}[language=make]
program = block "." .
block = [ "const" ident "=" number {"," ident "=" number} ";"]
        [ "var" ident {"," ident} ";"]
        { "procedure" ident ";" block ";" } statement .
statement = [ ident ":=" logical
              | "call" ident 
              | "read" ident
              | "write" expression 
              | "begin" statement {";" statement } "end" 
              | "if" condition "then" statement [ "else" statement ]
              | "goto" label
              | "execute" "0x" hexstring
              | "while" condition "do" statement ].
label = "label" ident.
condition = "odd" logical |
            logical ("="|"#"|"<"|"<="|">"|">=") logical .
logical = [ "not" ] expression
            | expression ("and"|"or"|"xor"|"left"|"right") logical.
expression = expression { ("+"|"-") term}
               | ["+"|"-"] term.
term = factor {("*"|"/") factor}.
factor = ident | number | "(" logical ")".
\end{lstlisting}
\end{landscape}
\clearpage
\section{Obr\'azky program\r{u}}\label{app:obr}
\begin{figure}[!ht]
	\centering
	\includegraphics[width=.8\textwidth]{../pictures/hilbert.png}
	\caption{V\'ystup programu \texttt{hilbert.pl0}.}
	\label{obr:hilbert}
\end{figure}
\begin{figure}[!ht]
	\centering
	\includegraphics[width=.8\textwidth]{../pictures/circle.png}
	\caption{V\'ystup programu \texttt{circle.pl0}.}
	\label{obr:circle}
\end{figure}
\clearpage
\end{document}
